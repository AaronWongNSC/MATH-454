%\title{Math 453 Getting Started}
\documentclass[addpoints]{exam}

\usepackage{amsmath,enumitem,wrapfig,url}
\usepackage{tikz}
\usepackage{hyperref}
\hypersetup{
    colorlinks = true,
    urlcolor = red!50!black}

\newcommand{\StudentName}{Student Name}
\newcommand{\AssignmentName}{Getting Started}

\pagestyle{headandfoot}
\runningheadrule
\firstpageheadrule
\firstpageheader{Math 453}{\StudentName}{\AssignmentName}
\runningheader{Math 453}{\StudentName}{\AssignmentName}
\firstpagefooter{}{}{}
\runningfooter{}{}{}

\printanswers

\begin{document}


Organize your work and show any work that you want credit for. Use full sentences where possible.

\begin{questions}

\question This course follows Specifications Grading. You can read a bit about Specs Grading \href{http://rtalbert.org/blog/2015/Specs-grading-report-part-1/}{here} and \href{http://rtalbert.org/blog/2015/Specs-grading-report-part-2/}{here}. Read these blog posts and look over our own \href{http://sergeballif.github.io/NSC-Math-453/Syllabus.html#to-earn-an-a}{grade Specifications}. Write a few paragraphs explaining Specifications Grading and how grades will be determined in our class. Decide on what grade you will be working towards and write out a plan of how you will attain that grade.

\begin{solution}
I can write beautiful, clear sentences that show my stellar communication skills.
\end{solution}

\question This course uses a method of instruction called Inquiry Based Learning (IBL). You can read about IBL  \href{http://www.inquirybasedlearning.org/?page=What_is_IBL}{here} and \href{https://www.artofmathematics.org/blogs/jfleron/what-is-inquiry-based-learning}{here}. Read these pages and look over the textbook. Write a few paragraphs explaining the idea of IBL and how it might be used effectively in our class. Include any concerns or helpful ideas.


\question What questions do you have about the course?


\end{questions}
\end{document}
